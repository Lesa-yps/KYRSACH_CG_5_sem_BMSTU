% Список использованных источников
\begin{center}
    \MakeUppercase{\large Список использованных источников}
\end{center}
\addcontentsline{toc}{section}{Список использованных источников} % Добавляем в оглавление

% Удаление "Список литературы" по умолчанию и получившихся пробелов
\renewcommand{\refname}{}
\vspace{-11mm}

\makeatletter
\renewcommand\@biblabel[1]{#1.} % Изменение формата номера
\makeatother

\begin{thebibliography}{9} 
    \bibitem{rodgers}
    Роджерс Д. Алгоритмические основы машинной графики. — М.: Мир, 1989. — 512 с.
    
    \bibitem{iter}
    ИТЕР. Модель системы защиты [Электронный ресурс]. — Режим доступа: \url{https://iter.ru/model_sfz.html#section_9}, дата обращения: 21.09.2024.
    
    \bibitem{polski}
    Польский С. В. Компьютерная графика: учеб.-метод. пособие. — М.: ГОУ ВПО МГУЛ, 2008. — 38 с.
    
    \bibitem{shishkin}
    Шикин Е. В., Боресков А. В. Компьютерная графика. Полигональные модели. — М.: Диалог-МИФИ, 2001. — 464 с.
    
    \bibitem{tyrlapov}
    Турлапов В. Е. Удаление невидимых поверхностей. Оптимизация. Тени [Электронный ресурс]. — Нижегородский государственный университет имени Н. И. Лобачевского, 2013. — Режим доступа: \url{http://www.graph.unn.ru/rus/materials/CG/CG13_HSROptimization.pdf}, дата обращения: 21.11.2024.
    
    \bibitem{zbuffer}
    Алгоритм, использующий z–буфер [Электронный ресурс]. — Режим доступа: \url{https://compgraph.tpu.ru/zbuffer.htm}, дата обращения: 21.11.2024.
    
    \bibitem{boreskov}
    Шикин Е. В., Боресков А. В. Компьютерная графика. Динамика, реалистические изображения. — М.: Диалог-МИФИ, 1995. — 288 с.
    
    \bibitem{kgau}
    Дерягина О. В., Корниенко В. В., Лагерь А. И., Скоробогатова Т. Е. Компьютерная графика: учебное пособие. — Красноярск: Красноярский государственный аграрный университет, 2014. — 322 с. — Режим доступа: \url{http://www.kgau.ru/distance/etf_06/komp-grafika/index.htm}, дата обращения: 21.11.2024.
    
    \bibitem{compgraphics} 
Компьютерная графика: теория, алгоритмы, примеры на C++ и OpenGL [Электронный ресурс] / Освещение и модель затенения. URL: \url{https://compgraphics.info/3D/lighting/shading_model.php} (дата обращения: 21.11.2024).

	\bibitem{romanyuk_kurinnyy}
Романюк А.Н., Куринный М.В. 
Алгоритмы построения теней // Компьютеры + программы. – 2000. – № 8–9. – С. 5–6.
    
\end{thebibliography}
