% Список использованных источников
\begin{center}
    \MakeUppercase{\large Список использованных источников}
\end{center}
\addcontentsline{toc}{section}{Список использованных источников} % Добавляем в оглавление

% Удаление "Список литературы" по умолчанию и получившихся пробелов
\renewcommand{\refname}{}
\vspace{-11mm}

\begin{thebibliography}{9}
    
    \bibitem{rodgers}
Д. Роджерс, \textit{Алгоритмические основы машинной графики}. Москва: Мир, 1989. --- 512 стр.

    \bibitem{iter}
Компания ИТЕР, "Модель системы защиты". Дата обращения: 21 сентября 2024 г. [Электронный ресурс]. Доступно по адресу: \url{https://iter.ru/model_sfz.html#section_9}

	\bibitem{polski}
Польский, С. В.
П53 Компьютерная графика : учебн.-методич. пособие. – М. : ГОУ
ВПО МГУЛ, 2008. – 38 с.

	\bibitem{shishkin}
Шикин Е.В., Боресков А.В. Компьютерная графика. Полигональные модели. –М.:
ДИАЛОГ-МИФИ, 2001.-464с.

	\bibitem{tyrlapov}
В. Е. Турлапов, \emph{Удаление невидимых поверхностей. Оптимизация. Тени}, Нижегородский государственный университет имени Н. И. Лобачевского, Национальный исследовательский университет, факультет вычислительной математики и кибернетики, кафедра программного обеспечения, курс: КГ 255. Компьютерная графика, 2013. Доступно по ссылке: \url{http://www.graph.unn.ru/rus/materials/CG/CG13_HSROptimization.pdf}.

    
\end{thebibliography}