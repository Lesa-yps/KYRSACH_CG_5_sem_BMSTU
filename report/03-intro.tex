% Введение
% Центрируем заголовок и делаем его капсом
\begin{center}
    \MakeUppercase{\large Введение}
\end{center}
\addcontentsline{toc}{section}{Введение} % Добавляем в оглавление

В настоящее время введение технологий 3D моделирования активно развивается в самых разных сферах: архитектуре, дизайне, инженерии, безопасности и оперативном реагировании. Программное обеспечение для построения 3D сцен позволяет визуализировать объекты и пространства, что существенно облегчает их анализ, планирование и эксплуатацию. В особенности, технологии моделирования помещений с гибкой настройкой планировок и объектов находят важное применение для задач охраны и контроля доступа на стратегически важные объекты.~\cite{iter}

Одной из ключевых сфер, в которой данная тема является особенно актуальной, выступает обеспечение безопасности и оперативного реагирования силовых структур на проникновение на охраняемые объекты. В условиях возросших угроз безопасности и повышенных требований к защите территорий и объектов, необходимость быстрого создания точных моделей помещений с целью их анализа является приоритетной задачей. Использование 3D моделей позволяет оперативно прорабатывать сценарии вторжений, рассчитывать оптимальные пути пресечения проникновений и создавать план эвакуации или нейтрализации угрозы.

Для силовых структур важно иметь инструменты, которые позволяют в режиме реального времени моделировать различные планировки и адаптировать стратегические планы по обеспечению безопасности. Программа, позволяющая моделировать помещения с возможностью добавления объектов (стен, окон, дверей), их перемещения и изменения, управления камерой, может стать незаменимым инструментом для подготовки к операциям, тренировок и планирования реагирования в критических ситуациях.

Таким образом, разработка программы построения 3D сцен помещений имеет важное практическое значение, поскольку позволяет оперативно создавать модели для анализа и разработки решений по обеспечению безопасности.

Целью данной курсовой работы является разработка программного обеспечения для создания и редактирования 3D сцен помещений с возможностью интерактивного добавления объектов (стен, окон, дверей), их перемещения, изменения, поворота, а также обеспечения сохранения и загрузки моделей.

\vspace{0.25cm}
В рамках работы были поставлены следующие задачи:

\begin{itemize}[label=---]
    \item проанализировать требования к программе и исследовать существующие решения;
    
    \item изучить алгоритмы реализации технических решений и выбрать наиболее подходящие для работы с 3D сценами;
    
    \item разработать архитектуру и реализовать программу;
    
    \item протестировать программу целиком и её отдельные модули;
    
    \item исследовать производительность программы;
    
    \item подготовить отчётную документацию.
\end{itemize}

\newpage