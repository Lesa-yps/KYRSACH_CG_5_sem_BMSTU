% Введение
% Центрируем заголовок и делаем его капсом
\begin{center}
    \MakeUppercase{\large Введение}
\end{center}
\addcontentsline{toc}{section}{Введение} % Добавляем в оглавление

В настоящее время введение технологий 3D моделирования активно развивается в самых разных сферах: архитектуре, дизайне, инженерии, безопасности и оперативном реагировании. Программное обеспечение для построения 3D сцен позволяет визуализировать объекты и пространства, что существенно облегчает их анализ, планирование и эксплуатацию. В особенности, технологии моделирования помещений с гибкой настройкой планировок и объектов находят важное применение для задач охраны и контроля доступа на стратегически важные объекты.

Одной из ключевых сфер, в которой данная тема является особенно актуальной, выступает обеспечение безопасности и оперативного реагирования силовых структур на проникновение на охраняемые объекты. В условиях возросших угроз безопасности и повышенных требований к защите территорий и объектов, необходимость быстрого создания точных моделей помещений с целью их анализа является приоритетной задачей. Использование 3D моделей позволяет оперативно прорабатывать сценарии вторжений, рассчитывать оптимальные пути пресечения проникновений и создавать план эвакуации или нейтрализации угрозы.

Для силовых структур важно иметь инструменты, которые позволяют в режиме реального времени моделировать различные планировки и адаптировать стратегические планы по обеспечению безопасности. Программа, позволяющая моделировать помещения с возможностью добавления объектов (стен, окон, дверей), их перемещения, масштабирования и управления камерой, может стать незаменимым инструментом для подготовки к операциям, тренировок и планирования реагирования в критических ситуациях.

Таким образом, разработка программы построения 3D сцен помещений имеет важное практическое значение, поскольку позволяет оперативно создавать модели для анализа и разработки решений по обеспечению безопасности.

Целью данной курсовой работы является разработка программного обеспечения для создания и редактирования 3D сцен помещений с возможностью интерактивного добавления объектов (стен, окон, дверей), их перемещения, масштабирования, поворота, а также обеспечения сохранения и загрузки моделей.

\vspace{0.25cm}
В рамках работы были поставлены следующие задачи:

\begin{enumerate}
    \item Анализ требований к программе и исследование существующих решений:
    \begin{enumerate}
        \item Изучить программные продукты для 3D моделирования помещений, чтобы понять их функциональные особенности и интерфейсные решения.
        \item Оценить, какие элементы и функции наиболее важны для конечного пользователя.
    \end{enumerate}
    
    \item Изучить алгоритмы реализации технических решений и выбрать наиболее подходящие для работы с 3D сценами.
    
    \item Разработка архитектуры программы:
    \begin{enumerate}
        \item Спроектировать структуру программы, определив основные компоненты: объекты сцены (стены, окна, двери), камера, управление сценой.
        \item Разработать систему хранения данных о 3D моделях, обеспечивающую сохранение и загрузку сцены.
    \end{enumerate}
    
    \item Реализация объектов для создания сцены:
    \begin{enumerate}
        \item Создать базовые 3D объекты (стена, окно, дверь) с параметрами (размеры, позиции, углы поворота, текстуры).
        \item Обеспечить возможность добавления, перемещения, удаления, масштабирования и поворота объектов на сцене.
    \end{enumerate}
    
    \item Разработка системы управления камерой:
    \begin{enumerate}
        \item Предоставить пользователю возможность управления камерой для осмотра сцены под разными углами.
        \item Реализовать функции перемещения камеры, вращения вокруг объектов, изменения масштаба.
    \end{enumerate}
    
    \item Реализация пользовательского интерфейса (UI):
    \begin{enumerate}
        \item Разработать удобный интерфейс для добавления и редактирования объектов сцены.
        \item Включить панели инструментов для выбора объектов, изменения их параметров, управления сценой и камерой.
    \end{enumerate}
    
    \item Сохранение и загрузка 3D сцен:
    \begin{enumerate}
        \item Реализовать функционал сохранения текущей сцены в файл в специальном формате, чтобы пользователи могли продолжить работу позже.
        \item Предусмотреть возможность загрузки ранее сохранённых сцен для редактирования.
    \end{enumerate}
    
    \item Тестирование программы:
    \begin{enumerate}
        \item Провести тестирование работы программы для различных вариантов планировки помещений.
        \item Проверить корректность работы с сохранением и загрузкой сцен, взаимодействие с объектами и камерой.
    \end{enumerate}
    
    \item Оценка производительности программы:
    \begin{enumerate}
        \item Провести анализ производительности программы при увеличении количества объектов на сцене.
        \item Оптимизировать работу с 3D объектами для плавного взаимодействия даже при больших сценах.
    \end{enumerate}
    
    \item Документирование и подготовка отчётной документации:
    \begin{enumerate}
        \item Описать процесс разработки, результаты тестирования, а также подготовить руководство пользователя для программы.
    \end{enumerate}
\end{enumerate}

\newpage