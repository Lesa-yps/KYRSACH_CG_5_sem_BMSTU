% Заключение
\begin{center}
    \MakeUppercase{\large Заключение}
\end{center}
\addcontentsline{toc}{section}{Заключение} % Добавляем в оглавление

В результате выполнения курсовой работы было разработано программное обеспечение для создания и редактирования 3D сцен помещений с возможностью интерактивного добавления объектов (стен, окон, дверей), их перемещения, масштабирования, поворота, а также обеспечения сохранения и загрузки моделей. Цель достигнута.

\vspace{0.25cm}
Выполнены все задачи:

\begin{itemize}[label=---]

	\item были проанализированы требования к программе и исследованы существующие решения;
    
    \item изучены алгоритмы реализации технических решений и выбраны наиболее подходящие для работы с 3D сценами;
    
    \item разработана архитектура и реализована программы;
    
    \item протестирована программа целиком и её отдельные модули;
    
    \item проведены исследования производительности программы;
    
    \item документирована и подготовлена отчётная документация.

\end{itemize}


В рамках работы была разработана программа для создания и редактирования 3D сцен помещений с возможностью добавления и управления объектами, а также работы с камерой. Проведённые исследования подтвердили эффективность использования алгоритма Z-буфера для удаления невидимых линий и построения теней, а также модели затенения по Фонгу для создания реалистичной визуализации.

Анализ производительности показал, что параллельная обработка значительно сокращает время выполнения задач, особенно при увеличении сложности сцены, что подтверждает целесообразность её применения.

Разработанное программное обеспечение обладает широкими возможностями и может быть использовано в задачах проектирования, анализа и безопасности. Перспективы развития включают добавление поддержки виртуальной реальности и оптимизацию для работы с более сложными сценами.

\newpage